% Created 2018-01-19 Fri 19:38
\documentclass[12pt,letterpaper]{article}
\usepackage{graphicx,rotating,overpic,eepic,amssymb,amsmath,color,wasysym,hyperref,soul}
%\usepackage[backend=biber,natbib=true,style=numeric,sorting=none,firstinits=true,maxbibnames=15]{biblatex}
\usepackage[numbers]{natbib} 
%\bibliography{libraryFixed}
\usepackage{longtable}
\usepackage{xspace}
\usepackage{float}% hello ;)
\newcommand{\commentt}[1]{}
\newcommand{\Rey}{\mbox{\textit{Re}}}  % Reynolds number
\newcommand{\ReyRBC}{\Rey_\mathrm{RBC}}  % Reynolds number
\newcommand{\mum}{$\mu$m\xspace}
\newcommand{\muL}{$\mu$L\xspace}
\newcommand{\capi}{\mathrm{Ca}}
\newcommand{\gdot}{\dot{\gamma}}
\newcommand{\etal}{et al.\,}
\newcommand{\invs}{~s$^{-1}$\xspace}
\usepackage[usenames,dvipsnames,svgnames,x11names,table]{xcolor}
%\usepackage[latin1]{inputenc}
\usepackage{tikz}
\usetikzlibrary{shapes,arrows}

%\usepackage{psfrag,pstricks}
%\addtolength{\oddsidemargin}{-.675in}
%\addtolength{\evensidemargin}{-.675in}
%\addtolength{\textwidth}{1.35in}

\addtolength{\topmargin}{-.675in}
\addtolength{\textheight}{1.35in}
\topmargin -1.5cm        % read Lamport p.163
\oddsidemargin -0.04cm   % read Lamport p.163
\evensidemargin -0.04cm  % same as oddsidemargin but for left-hand pages
\textwidth 16.59cm
\textheight 21.94cm 
%\pagestyle{empty}       % Uncomment if don't want page numbers
\parskip 7.2pt           % sets spacing between paragraphs
\renewcommand{\baselinestretch}{1.0} 	% Uncomment for 1.5 spacing between lines
\parindent 0pt		  % sets leading space for paragraphs
\usepackage[utf8]{inputenc}
\usepackage[T1]{fontenc}
\usepackage{fixltx2e}
\usepackage[font=small,labelfont=bf,margin=10pt]{caption}
\hypersetup{
  pdfborderstyle={/S/U/W 1},
  pdfborder={0 0 100},
  pdfkeywords={},
  pdfsubject={},
  pdfcreator={Emacs 24.1.1 (Org mode 8.1.2)}} %%%%%%%%%%%%%%%%%%%%%%%%%%%%%%%%%%%%%
\usepackage[utf8]{inputenc}
\usepackage[T1]{fontenc}
\usepackage{fixltx2e}
\usepackage{graphicx}
\usepackage{longtable}
\usepackage{float}
\usepackage{rotating}
\usepackage[normalem]{ulem}
\usepackage{amsmath}
\usepackage{textcomp}
\usepackage{marvosym}
\usepackage{wasysym}
\usepackage{amssymb}
\usepackage{hyperref}
\tolerance=1000
% To change the background color of verbatim sections in latex
\colorlet{LightSteelBlue10}{LightSteelBlue1!50}
\colorlet{SteelBlue40}{SteelBlue4!60!black}
\newcommand{\verbStyle}[1]{{\color{SteelBlue40}\colorbox{LightSteelBlue10}{{#1}}}}
\let\OldTexttt\texttt
\renewcommand{\texttt}[1]{\OldTexttt{\verbStyle{#1}}}
\author{Yu-Ping Yang}
\date{December 20, 2017}
\title{Instructions}
\hypersetup{
 pdfauthor={Yu-Ping Yang},
 pdftitle={Instructions},
 pdfkeywords={},
 pdfsubject={},
 pdfcreator={Emacs 24.3.1 (Org mode 8.3.1)}, 
 pdflang={English}}
\begin{document}

\maketitle
\begin{itemize}
\item The following five files will be output from the interface (under \texttt{inputs/} directory):
\begin{itemize}
\item \texttt{eweld.in}
\item \texttt{eweld\_weld\_parameters.in}
\item \texttt{eweld\_boundary\_condition.in}
\item \texttt{eweld\_preheat\_interpass\_temperature.in}
\item \texttt{eweld\_temperature\_monitor.in}
\item \texttt{eweld\_mesh\_key.txt} (Not need to do now. This option will allow users to input their own meshes. )
\end{itemize}

\item For automatic mesh, the following steps will be run:
\begin{enumerate}
\item Check if \texttt{pass\_coordinates.out} exists in \texttt{input} directory, if no, run \\
\texttt{determine\_passes\_arc\_v4.exe} to create \texttt{pass\_coordinates.out}\footnote{On linux, compile      \texttt{determine\_passes\_arc\_v4.out}, to get \texttt{determine\_passes\_arc\_v4.out} via \texttt{gfortran determine\_passes\_arc\_v4.for -o determine\_passes\_arc\_v4.out}}. 
\texttt{eweld.in} will be input.
\item Run \texttt{Automesh\_v14.py} with SALOME to create \texttt{Mesh\_3D.unv}
\begin{enumerate}
\item The files will be input:
\begin{itemize}
\item \texttt{./inputs/eweld.in}
\item \texttt{./inputs/eweld\_weld\_parameters.in}
\item \texttt{./setting/Setting\_arc\_efficiency\_dfault.in}
\end{itemize}
\item The files will be output: 
\begin{itemize}
\item \texttt{Mesh\_3D.unv}
\item \texttt{model\_dflux.for}
\item \texttt{model\_step.in}
\end{itemize}
\end{enumerate}
\item Run 
\begin{verbatim}
python2 tools/unv2calculix.py Mesh_3D.unv Model3d
\end{verbatim}
\texttt{Model3d.inp} will be created.
\item To generate the \texttt{model\_film.in} file (using cgx and unical), run:
\begin{verbatim}
./createFilm.sh
\end{verbatim}
\item Run 
\begin{verbatim}
python Analysis_file_create.py
\end{verbatim}
\begin{itemize}
\item The files will be input:
\begin{itemize}
\item \texttt{./inputs/eweld.in}
\item \texttt{eweld\_boundary\_condition.in}
\item \texttt{eweld\_preheat\_interpass\_temperature.in}
\end{itemize}
\item The files will be output:
\begin{itemize}
\item \texttt{model\_bc.in}
\item \texttt{model\_ele4.in}
\item \texttt{model\_ele6.in}
\item \texttt{model\_ele8.in}
\item \texttt{model\_film.in}
\item \texttt{model\_group.in}
\item \texttt{model\_ini\_temperature.in}
\item \texttt{model\_material.in}
\item \texttt{model\_node.in}
\end{itemize}
\end{itemize}
\item Move \texttt{model\_dflux.for} to the Calculix directory and rename to \texttt{dflux.f}, and compile CalculiX
\item Run \texttt{analysis.inp} with calculix
\end{enumerate}
\end{itemize}
\end{document}
